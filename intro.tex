\section{Introduction}
\label{sec:intro}

With the increasing of sensor performance and the rapid expansion of social network data, it is necessary to parallelize or distribute the analysis and computation of large scale data. However, the parallelization of algorithms requires programmers to have a rich knowledge of distributed systems. A large number of big data analysis systems have been proposed which provide high-level language for expressing the computation logic \cite{}. Even so, these systems still require the users to understand the specific programming logic. For example, the typical representative of the BSP computing model, Pregel \cite{} adopts a vertex-centric programming model. It requires programmers to write vertex-centric programs and to manage the communication based on the algorithm. 

%Maiter[$]is an asynchronous graph computing system which proposed a delta-based accumulative iterative computation method. Maiter provide an asynchronous programming model, demand the programmer rewrite the algorithm in a DAIC way. The complexity of parallize or distribute an algorithm wasn��t significantly reduced.

Datalog is a well known recursive query language. It is convenient to express large scale data analysis algorithms with its high-level declarative semantics. In recent years, many system research efforts have raised to improve the performance and scalability of Datalog systems. Socialite \cite{socialite} provides a large scale graph evaluation system supporting both sequential and distribute environment. BigDatalog \cite{Bigdatalog} is built on Spark which can efficiently support recursion computation. Myria \cite{MyriaX} first proposed the monotonic asynchronous aggregation model and implement a datalog System on share-nothing engines based on myria
%[$Myria],a big data management system.[���������ص�ǿ���첽] It is worth mentioning that MyriaX is the first work that support recursive program with aggregation.
\textcolor{red}{This paragraph should be improved. say more about datalog, say more about its history, it was proposed 30 years ago but only become a little more popular after 2010 due to the graph/social network processing requirement. say sth about its disadvantages and advantages. then use another paragraph to introduce recent datalog systems}



Compared with synchronous recursive processing, asynchronous recursive processing has many advantages, such as fast convergence \cite{}, more efficient resource utilization \cite{}, and priority scheduling \cite{}. However, in practical usage, synchronous models are more popular than asynchronous models, which can be attributed to the following three points:

First, \textbf{Non-guaranteed Correctness}. There are several prior works that have employed asynchronous processing engine for improving their system performance. However, these systems only implement asynchronous execution of parallel/distributed algorithms, without any correctness guarantee of the results \cite{groute, graphlab,...}. Asynchronous computation model are blindly used and may result in inconsistent results for convex functions. Maiter \cite{} provides the sufficient conditions for correct asynchronous computations, but these conditions are only suitable in the context of vertex-centric graph computations. 

Second, \textbf{More Complexity}. Programmers are used to write sequential programs or synchronous parallel programs. Had the asynchronization conditions been formally given, it is still hard for a non-expert programmer to manually verify these conditions from their programs. In addition, writing asynchronous programs and designing asynchronous systems are even harder, because asynchronous implies disorganized and as a result complicated. Experience from Google \cite{} strongly suggests a synchronous programming model, since asynchronous code is a lot harder to write, tune, and debug. 

Third, \textbf{Unstable Performance}. Asynchronous iterative processing avoids the intermediate result coordination phase. The parallel executions of operations are not synchronized and not strictly ordered. This implies that the computations and communications are not under control any more, which may lead to stale computations/communications and potentially reduces the efficiency. The performance gain from asynchronous computation may be not enough to compensate for the performance loss from stale computations/commmunications, leading to unstable performance.


In order to solve these problems��we design and implement a datalog system supporting asynchronous execution, A3LOG. Compared with previous works, our system is able to check whether an algorithm can be correctly asynchronized by a novel Sufficient condition and then choose the better execution model. Such that Asynchronous calculations needn��t to be used blindly. So the system can greatly reduce the workload of the programmer. In order to meet different use requirements, the system provides both  shared-memory runtime engine and distributed runtime engine. The contribution are summarized as follow 